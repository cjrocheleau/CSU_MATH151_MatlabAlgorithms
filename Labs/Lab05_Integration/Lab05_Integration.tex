% Don't touch this %%%%%%%%%%%%%%%%%%%%%%%%%%%%%%%%%%%%%%%%%%%
\documentclass[11pt]{article}
\usepackage{fullpage}
\usepackage[left=1in,top=1in,right=1in,bottom=1in,headheight=3ex,headsep=3ex]{geometry}
\usepackage{graphicx}
\usepackage{float}

\newcommand{\blankline}{\quad\pagebreak[2]}
\newcommand{\comment}[1]{}

\usepackage{setspace}
\usepackage{multicol}
%\usepackage{indentfirst}
\usepackage{fancyhdr,lastpage}
\usepackage{url}
\pagestyle{fancy}
\usepackage{hyperref}
\usepackage{lastpage}
\usepackage{amsmath}
\usepackage{layout}
\usepackage{enumitem}

\usepackage{array}
\usepackage{booktabs}

\lhead{}
\chead{}
\rhead{\footnotesize MATH-151 Mathematical Algorithms in Matlab - Fall 2023}

\lfoot{}
\cfoot{\small \thepage/\pageref*{LastPage}}
\rfoot{}

\usepackage{array, xcolor}
\usepackage{color,hyperref}
\hypersetup{colorlinks,breaklinks,linkcolor=blue,urlcolor=blue,anchorcolor=blue,citecolor=black}

%%%%%%%%%%%%%%%%%%%%%%%%%%%%%%%%%%%%%%%%%%%%%%%%%%%%%
%%%%%%%%%%%%%%%%%%%%%%%%%%%%%%%%%%%%%%%%%%%%%%%%%%%%%
%%%%%%%%%%%%%%%%%%%%%%%%%%%%%%%%%%%%%%%%%%%%%%%%%%%%%

\begin{document}
	
	\begin{center}
		\Large{\textbf{MATH-151 Lab 5: Integration Methods}}\\
			\medskip
		\normalsize{\textbf{Due:} Monday, October 2, 2023, 10:00am} 
	\end{center}
	\noindent\makebox[\linewidth]{\rule{\textwidth}{0.4pt}}
	%%%%%%%%%%%%%%%%%%%%%%%%%%%%%%%%%%%%%%%%%%%%%%%%%%%%%
	Please perform the following tasks using Matlab, submitting all relevant code. You are welcomed to work with other students, however each student must submit their own unique code.
	%%%%%%%%%%%%%%%%%%%%%%%%%%%%%%%%%%%%%%%%%%%%%%%%%%%%%	
	\section*{Task 1: Integrating Data}
	\noindent In many cases, we don't have a known function for the quantities we are trying to integrate, we just have a few data points and try to make the best with that. For example, many GPS trackers will integrate speed measurements to try to ``guess" where you are going to be located next. This is almost never exactly correct, but it can give us valuable information. For this task, suppose we are given three measurements $(0,0)$, $(0.5,0.3256)$, and $(1,-0.1108)$, lets use our integration methods to get an estimate of the integral of this function between 0 and 1.
	\begin{enumerate}[label=\alph*)]
		\item Using the rectangular rule of your choice, calculate an estimate of this integral.
		\item Now, try trapezoid rule. What is your estimate now?
		\item Finally, find an estimate of this integral using Simpson's rule.
		\item Based on your findings from part a through c, what is your best guess for the integral of this function?
	\end{enumerate}
	%%%%%%%%%%%%%%%%%%%%%%%%%%%%%%%%%%%%%%%%%%%%%%%%%%%%%
	\section*{Task 2: Integration Flexibility}
	\noindent When we do have a defined function, we have some ``knobs" we can turn (different methods, step sizes) in order to make our approximation as accurate as desired. This task is to explore our options in numerical integration in order to increase our accuracy.\\
	\begin{enumerate}[label=\alph*)]
		\item The Weibull distribution is a commonly used probability distribution in reliability analysis. For this example, suppose we have a Weibull distribution given by 
		\begin{equation*}
			\displaystyle f(x) = \frac{k}{\lambda}\Big(\frac{x}{\lambda}\Big)^{k-1}e^{-(\frac{x}{\lambda})^k}
		\end{equation*}
		where $k=\frac{3}{2}$ and $\lambda = \frac{3}{2}$. Using this distribution we can find the probability this product fails in 2 years is 
		\begin{equation*}
			\int_0^2f(x)dx = 0.785533282932306\dots
		\end{equation*}
		Write a numerical integration scheme that computes this probability correctly to 4 significant digits.
	\end{enumerate}

	%%%%%%%%%%%%%%%%%%%%%%%%%%%%%%%%%%%%%%%%%%%%%%%%%%%%%
	\begin{center}
		\vfill
		\textit{Remember, vectorizing functions is usually a good idea!}
	\end{center}
\end{document}