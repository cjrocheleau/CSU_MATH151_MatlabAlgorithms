% Don't touch this %%%%%%%%%%%%%%%%%%%%%%%%%%%%%%%%%%%%%%%%%%%
\documentclass[11pt]{article}
\usepackage{fullpage}
\usepackage[left=1in,top=1in,right=1in,bottom=1in,headheight=3ex,headsep=3ex]{geometry}
\usepackage{graphicx}
\usepackage{float}

\newcommand{\blankline}{\quad\pagebreak[2]}
\newcommand{\comment}[1]{}

\usepackage{setspace}
\usepackage{multicol}
%\usepackage{indentfirst}
\usepackage{fancyhdr,lastpage}
\usepackage{url}
\pagestyle{fancy}
\usepackage{hyperref}
\usepackage{lastpage}
\usepackage{amsmath}
\usepackage{layout}
\usepackage{enumitem}

\usepackage{array}
\usepackage{booktabs}

\lhead{}
\chead{}
\rhead{\footnotesize MATH-151 Mathematical Algorithms in Matlab - Fall 2023}

\lfoot{}
\cfoot{\small \thepage/\pageref*{LastPage}}
\rfoot{}

\usepackage{array, xcolor}
\usepackage{color,hyperref}
\hypersetup{colorlinks,breaklinks,linkcolor=blue,urlcolor=blue,anchorcolor=blue,citecolor=black}

%%%%%%%%%%%%%%%%%%%%%%%%%%%%%%%%%%%%%%%%%%%%%%%%%%%%%
%%%%%%%%%%%%%%%%%%%%%%%%%%%%%%%%%%%%%%%%%%%%%%%%%%%%%
%%%%%%%%%%%%%%%%%%%%%%%%%%%%%%%%%%%%%%%%%%%%%%%%%%%%%

\begin{document}
	
	\begin{center}
		\Large{\textbf{MATH-151 Lab 3: Functions and Recursion}}\\
			\medskip
		\normalsize{\textbf{Due:} Monday, September 18, 2023, 10:00am} 
	\end{center}
	\noindent\makebox[\linewidth]{\rule{\textwidth}{0.4pt}}
	%%%%%%%%%%%%%%%%%%%%%%%%%%%%%%%%%%%%%%%%%%%%%%%%%%%%%
	Please perform the following tasks using Matlab, submitting all relevant code. You are welcomed to work with other students, however each student must submit their own unique code.
	%%%%%%%%%%%%%%%%%%%%%%%%%%%%%%%%%%%%%%%%%%%%%%%%%%%%%
	\section*{Task 1: Return of the Factorials}
	\noindent We looked at the factorial function previously to practice using loops in Lab 1. The same property to compute the factorial using a loop also makes it very well-suited to be used as a recursive function! As a reminder, the property of interest is below\\
	\begin{equation*}
		n! = n\times(n-1)!\text{, with base case } 0! = 1
	\end{equation*}
	\begin{enumerate}[label=\alph*)]
		\item Using this property, create a recursive function \texttt{recursive\_factorial(n)} that computes the factorial for positive integer \texttt{n}. Using this function, what is $16!$ ?
	\end{enumerate}
	%%%%%%%%%%%%%%%%%%%%%%%%%%%%%%%%%%%%%%%%%%%%%%%%%%%%%	
	\section*{Task 2: What's Your Sine?}
	\noindent Anyone who has taken a class with me should be familiar that I believe the Taylor series is one of the most important concepts in mathematics. In this task I will try to drive that concept home by showing you how it is used to approximate the Sine function. Reminder, the Taylor series for Sine using $N$ terms is given as 
	\begin{equation*}
		\sin(x) \approx \sum_{n=1}^{N}\frac{(-1)^{n-1} x^{2n-1}}{(2n-1)!}
	\end{equation*}
	\begin{enumerate}[label=\alph*)]
		\item Guess how many terms of the Taylor series Matlab uses for the \texttt{sin(x)} function.
		\item Create a function \texttt{taylor\_sine(x,N)} that accepts a vector \texttt{x} and an integer \texttt{N} and outputs a vector containing estimates of the Sine of each element of \texttt{x} using \texttt{N} terms of the Taylor series. Using \texttt{x = -pi:0.1:pi}, plot the outputs of \texttt{sin(x)} and \texttt{taylor\_sine(x,3)} as functions of \texttt{x} on the same axis. How well do they agree with each other? \textit{(Note: I would prefer if you use your factorial function from Task 1, but you may also use Matlab's built-in \texttt{factorial} function.)}
		\item Now try using \texttt{taylor\_sine(x,5)}, how does that look?
	\end{enumerate}
	

	%%%%%%%%%%%%%%%%%%%%%%%%%%%%%%%%%%%%%%%%%%%%%%%%%%%%%
	\begin{center}
		\vfill
		\textit{It can be helpful to test your functions with small, known examples first!}
	\end{center}
\end{document}