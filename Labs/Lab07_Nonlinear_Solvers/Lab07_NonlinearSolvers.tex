% Don't touch this %%%%%%%%%%%%%%%%%%%%%%%%%%%%%%%%%%%%%%%%%%%
\documentclass[11pt]{article}
\usepackage{fullpage}
\usepackage[left=1in,top=1in,right=1in,bottom=1in,headheight=3ex,headsep=3ex]{geometry}
\usepackage{graphicx}
\usepackage{float}

\newcommand{\blankline}{\quad\pagebreak[2]}
\newcommand{\comment}[1]{}

\usepackage{setspace}
\usepackage{multicol}
%\usepackage{indentfirst}
\usepackage{fancyhdr,lastpage}
\usepackage{url}
\pagestyle{fancy}
\usepackage{hyperref}
\usepackage{lastpage}
\usepackage{amsmath}
\usepackage{layout}
\usepackage{enumitem}

\usepackage{array}
\usepackage{booktabs}

\lhead{}
\chead{}
\rhead{\footnotesize MATH-151 Mathematical Algorithms in Matlab - Fall 2023}

\lfoot{}
\cfoot{\small \thepage/\pageref*{LastPage}}
\rfoot{}

\usepackage{array, xcolor}
\usepackage{color,hyperref}
\hypersetup{colorlinks,breaklinks,linkcolor=blue,urlcolor=blue,anchorcolor=blue,citecolor=black}

%%%%%%%%%%%%%%%%%%%%%%%%%%%%%%%%%%%%%%%%%%%%%%%%%%%%%
%%%%%%%%%%%%%%%%%%%%%%%%%%%%%%%%%%%%%%%%%%%%%%%%%%%%%
%%%%%%%%%%%%%%%%%%%%%%%%%%%%%%%%%%%%%%%%%%%%%%%%%%%%%

\begin{document}
	
	\begin{center}
		\Large{\textbf{MATH-151 Lab 7: Nonlinear Solvers}}\\
			\medskip
		\normalsize{\textbf{Due:} Monday, October 16, 2023, 10:00am} 
	\end{center}
	\noindent\makebox[\linewidth]{\rule{\textwidth}{0.4pt}}
	%%%%%%%%%%%%%%%%%%%%%%%%%%%%%%%%%%%%%%%%%%%%%%%%%%%%%
	Please perform the following tasks using Matlab, submitting all relevant code. You are welcomed to work with other students, however each student must submit their own unique code.
	
	%%%%%%%%%%%%%%%%%%%%%%%%%%%%%%%%%%%%%%%%%%%%%%%%%%%%%
	\section*{Task 1: Reaching the Bottom}
	\noindent Nonlinear Solvers are great at helping us find the extreme points of a function. For this task we will try to find the minimum of $f(x) = x\cos(x)$ for $-\pi \le x\le 2\pi$
	\begin{enumerate}[label=\alph*)]
		\item There are a lot of extreme points for this function, plot $f(x)$ with respect to $x$ and get a rough idea of two points $x_L$ and $x_R$ so that our minimum will be between those points.
		\item We know the minimum will occur when $f'(x) = 0$. We can find $f'(x) = \cos(x) - x\sin(x)$, using bisection method find the point $x^*$ between $x_L$ and $x_R$ for which $f'(x^*) = 0$. This should give us where our function is at a minimum.
		\item Plot $(x^*, f(x^*))$ as a red circle on your plot from part (a). Is this the minimum of the function?
	\end{enumerate}
	
	%%%%%%%%%%%%%%%%%%%%%%%%%%%%%%%%%%%%%%%%%%%%%%%%%%%%%	
	\section*{Task 2: See You Next Fall}
	\noindent We can find the vertical distance a parachuter will travel in $t$ seconds using the equation
	\begin{equation*}
		y(t) = \log(\cosh(t\sqrt{gk}))/k
	\end{equation*}
	Where $g=9.8m/s^2$ is the gravitational acceleration and $k=0.0034m^{-1}$ is a constant related to air resistance. We will use our nonlinear methods to solve how long it takes for the parachuter to fall 2000 meters, or find $t$ such that $y(t)-2000 = 0$.\\
	\textit{(Note: That is not a typo. $\cosh(x)$ is the hyperbolic cosine function, and is a built-in function in Matlab \texttt{cosh(x)}, similarly with \texttt{tanh(x)} in part c))}
	
	
	\begin{enumerate}[label=\alph*)]
		\item How long do you think it will take to fall 2000 meters?
		\item Using $t_0=0$ and $t_1$ as your initial guess from part (a), use Secant method to find how long it takes to fall 2000 meters to within 0.001 seconds. How many times did you need to loop before converging?
		\item We can find 
		\begin{equation*}
			y'(t) = \sqrt{\frac{g}{k}}\tanh(t\sqrt{gk})
		\end{equation*}
		Now use Newton's method to find how long it takes to fall 2000 meters to within 0.001 seconds. How many times did you need to loop using Newton's method before converging?
	\end{enumerate}

	%%%%%%%%%%%%%%%%%%%%%%%%%%%%%%%%%%%%%%%%%%%%%%%%%%%%%
	\begin{center}
		\vfill
		\textit{Be careful with the order of updating your $x_n$ values}
	\end{center}
\end{document}