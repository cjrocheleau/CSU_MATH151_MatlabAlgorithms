% Don't touch this %%%%%%%%%%%%%%%%%%%%%%%%%%%%%%%%%%%%%%%%%%%
\documentclass[11pt]{article}
\usepackage{fullpage}
\usepackage[left=1in,top=1in,right=1in,bottom=1in,headheight=3ex,headsep=3ex]{geometry}
\usepackage{graphicx}
\usepackage{float}

\newcommand{\blankline}{\quad\pagebreak[2]}
\newcommand{\comment}[1]{}

\usepackage{setspace}
\usepackage{multicol}
%\usepackage{indentfirst}
\usepackage{fancyhdr,lastpage}
\usepackage{url}
\pagestyle{fancy}
\usepackage{hyperref}
\usepackage{lastpage}
\usepackage{amsmath}
\usepackage{layout}
\usepackage{enumitem}

\usepackage{array}
\usepackage{booktabs}

\lhead{}
\chead{}
\rhead{\footnotesize MATH-151 Mathematical Algorithms in Matlab - Fall 2023}

\lfoot{}
\cfoot{\small \thepage/\pageref*{LastPage}}
\rfoot{}

\usepackage{array, xcolor}
\usepackage{color,hyperref}
\hypersetup{colorlinks,breaklinks,linkcolor=blue,urlcolor=blue,anchorcolor=blue,citecolor=black}

%%%%%%%%%%%%%%%%%%%%%%%%%%%%%%%%%%%%%%%%%%%%%%%%%%%%%
%%%%%%%%%%%%%%%%%%%%%%%%%%%%%%%%%%%%%%%%%%%%%%%%%%%%%
%%%%%%%%%%%%%%%%%%%%%%%%%%%%%%%%%%%%%%%%%%%%%%%%%%%%%

\begin{document}
	
	\begin{center}
		\Large{\textbf{MATH-151 Lab 2: Vectors and Plots}}\\
			\medskip
		\normalsize{\textbf{Due:} Monday, September 11, 2023, 10:00am} 
	\end{center}
	\noindent\makebox[\linewidth]{\rule{\textwidth}{0.4pt}}
	%%%%%%%%%%%%%%%%%%%%%%%%%%%%%%%%%%%%%%%%%%%%%%%%%%%%%
	Please perform the following tasks using Matlab, submitting all relevant code. You are welcomed to work with other students, however each student must submit their own unique code.
	%%%%%%%%%%%%%%%%%%%%%%%%%%%%%%%%%%%%%%%%%%%%%%%%%%%%%	
	\section*{Task 1: Moving Average}
	\noindent In any measurement system there will be measurement error. In some cases this error can introduce ``noise" into our measurements that distracts from the underlying information. One very common way to address measurement error is to use a moving average. We will look at how these work using an example of the vertical position of an object thrown into the air.\\
	\begin{enumerate}[label=\alph*)]
		\item First, we need to calculate the ``ground truth" or true locations of the object. Suppose the height of the object at time $t$ is given by $y(t) = 450 + 30t - 9.8t^2$. Generate a vector of times from 0 to 10s with a step size of 0.1s and use that vector to compute a vector of the object's height. Create a plot showing the height of the object over time.
		\item Next we need to add our measurement noise. Create vector of measured locations by taking the true locations and adding on some measurement error, $\tilde{y}(t) = y(t) + \epsilon$. For $\epsilon$ use \texttt{25*randn(size(y))}. (This generates a vector the same size as \texttt{y} using a normal distribution with a standard deviation of 25.) Plot your measurements as dots with no connecting line on the same plot as your true locations.
		\item We will ``smooth" out this measurement noise using a 3-point moving average. To do this, make a new vector using the following rule \textit{(Hint: this may require a loop)}
		\begin{equation*}
			\bar{y}(t_n) =  \begin{cases}
				\tilde{y}(t_n) & \text{if } n=1 \\
				\frac{\tilde{y}(t_n) + \tilde{y}(t_{n-1})}{2} & \text{if } n=2 \\
				\frac{\tilde{y}(t_n) + \tilde{y}(t_{n-1}) +\tilde{y}(t_{n-2})}{3} & \text{otherwise}
			\end{cases}
		\end{equation*}
		Calculate and plot this moving average as a black dotted line with your ground truth and measurements.
		\item Label your plot appropriately. At a minimum, this should include labeling the x and y axes, giving a title, and a legend.
	\end{enumerate}

	%%%%%%%%%%%%%%%%%%%%%%%%%%%%%%%%%%%%%%%%%%%%%%%%%%%%%
	\begin{center}
		\vfill
		\textit{Remember to use \texttt{hold on} to keep plots on the same axis!}
	\end{center}
\end{document}