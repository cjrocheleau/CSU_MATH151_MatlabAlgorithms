% Don't touch this %%%%%%%%%%%%%%%%%%%%%%%%%%%%%%%%%%%%%%%%%%%
\documentclass[11pt]{article}
\usepackage{fullpage}
\usepackage[left=1in,top=1in,right=1in,bottom=1in,headheight=3ex,headsep=3ex]{geometry}
\usepackage{graphicx}
\usepackage{float}

\newcommand{\blankline}{\quad\pagebreak[2]}
\newcommand{\comment}[1]{}

\usepackage{setspace}
\usepackage{multicol}
%\usepackage{indentfirst}
\usepackage{fancyhdr,lastpage}
\usepackage{url}
\pagestyle{fancy}
\usepackage{hyperref}
\usepackage{lastpage}
\usepackage{amsmath}
\usepackage{layout}
\usepackage{enumitem}

\usepackage{array}
\usepackage{booktabs}

\lhead{}
\chead{}
\rhead{\footnotesize MATH-151 Mathematical Algorithms in Matlab - Fall 2023}

\lfoot{}
\cfoot{\small \thepage/\pageref*{LastPage}}
\rfoot{}

\usepackage{array, xcolor}
\usepackage{color,hyperref}
\hypersetup{colorlinks,breaklinks,linkcolor=blue,urlcolor=blue,anchorcolor=blue,citecolor=black}

%%%%%%%%%%%%%%%%%%%%%%%%%%%%%%%%%%%%%%%%%%%%%%%%%%%%%
%%%%%%%%%%%%%%%%%%%%%%%%%%%%%%%%%%%%%%%%%%%%%%%%%%%%%
%%%%%%%%%%%%%%%%%%%%%%%%%%%%%%%%%%%%%%%%%%%%%%%%%%%%%

\begin{document}
	
	\begin{center}
		\Large{\textbf{MATH-151 Lab 6: Differentiation Methods}}\\
			\medskip
		\normalsize{\textbf{Due:} Monday, October 9, 2023, 10:00am} 
	\end{center}
	\noindent\makebox[\linewidth]{\rule{\textwidth}{0.4pt}}
	%%%%%%%%%%%%%%%%%%%%%%%%%%%%%%%%%%%%%%%%%%%%%%%%%%%%%
	Please perform the following tasks using Matlab, submitting all relevant code. You are welcomed to work with other students, however each student must submit their own unique code.
	%%%%%%%%%%%%%%%%%%%%%%%%%%%%%%%%%%%%%%%%%%%%%%%%%%%%%	
	\section*{Task 1: A Speeding Object!}
	\noindent Similar to what we saw last week, it is very common to use these methods with real world measurements to estimate derivatives without having a known function. For this problem, please download Lab06\_Data.mat from the course Canvas page and make sure it is located in the same folder as your script. You can load this data into Matlab using \texttt{load('Lab06\_Data.mat')}, this should add vectors \texttt{time} and \texttt{pos} to your workspace. 
	\begin{enumerate}[label=\alph*)]
		\item Using the central difference method, calculate and plot your estimate of this object's speed. \textit{(Note: Because the central difference method needs your $x$ value to be be in the center you may use the appropriate finite difference methods for the first and last values)}
		\item Say we are also interested in this object's acceleration over time. Calculate and plot an estimate of the second derivative from this data.
	\end{enumerate}
	%%%%%%%%%%%%%%%%%%%%%%%%%%%%%%%%%%%%%%%%%%%%%%%%%%%%%
	\section*{Task 2: Method Comparison}
	\noindent We learned that finite difference and central difference methods of numerical differentiation have differing orders of error. Hopefully, this makes some sense mathematically but lets take a second to look at what this means in an example. We will consider the function \\
	\begin{equation*}
		f(x) = \sin\Big(\frac{x}{3}\Big)^2
	\end{equation*}
	\begin{center}
		with derivative
	\end{center}	
	\begin{equation*}
		f'(x) = \frac{2}{3}\bigg(\sin\Big(\frac{x}{3}\Big)\cos\Big(\frac{x}{3}\Big)\bigg)
	\end{equation*}
	\begin{enumerate}[label=\alph*)]
		\item Use finite difference method to compute an approximation of $f'(x)$ from $-\pi$ to $\pi$ using a step size of 0.5. Plot this approximation on the same axis as the true $f'(x)$ function given above.
		\item Now approximate $f'(x)$ using the same $x$ grid using the central difference method. Plot this function on the same axis used in part (a). How do these approximations compare? Which looks more accurate. 
		\item In a new figure, plot the errors (true values - estimated values) for each of these methods. Is there a difference between these methods? Which is more accurate? 
	\end{enumerate}

	%%%%%%%%%%%%%%%%%%%%%%%%%%%%%%%%%%%%%%%%%%%%%%%%%%%%%
	\begin{center}
		\vfill
		\textit{Remember to label your plots correctly!}
	\end{center}
\end{document}