{}\documentclass[letterpaper,
compress,
xcolor=x11names,
%draft,
]{beamer}
% Package imports
\usepackage{mathtools} % imports `amsmath'
\DeclareMathOperator{\sech}{sech}
\usepackage{amssymb}
\usepackage{fixltx2e}
\usepackage{lmodern}
\usepackage{movie15}
%\usepackage{media9}
\usepackage{microtype}
\usepackage{animate}
\usepackage{subcaption}
\captionsetup{compatibility=false}

% I just did this
\usepackage[english]{babel}
\usepackage[utf8]{inputenc}
\usepackage{amsmath}
\usepackage{graphicx}
\usepackage[colorinlistoftodos]{todonotes}
\usepackage{tikz}
\usetikzlibrary{tikzmark}
\usepackage{array}
\usepackage{layout}
\usepackage{multicol}
\usepackage{multirow}
\usepackage{booktabs}
%I just did this

% `beamer' configuration
\usefonttheme{professionalfonts}
\useoutertheme[subsection=false,]{miniframes}
\setbeamercolor*{alerted text}{fg=red}
\setbeamercolor*{example text}{fg=black}
\definecolor{CSU_green}{RGB}{30, 70, 43}
\definecolor{CSU_gold}{RGB}{200, 195, 114}
\setbeamercolor*{lower separation line head}{bg=CSU_gold}
\setbeamercolor*{section in head/foot}{fg=white,bg=CSU_green}
\setbeamercolor*{subsection in head/foot}{bg=white}
\setbeamercolor*{upper separation line head}{bg=CSU_gold}
\setbeamercolor*{page number in head/foot}{fg=CSU_green}
\setbeamercolor*{normal text}{fg=black,bg=white}
\setbeamercolor*{palette tertiary}{fg=black,bg=black!10}
\setbeamercolor*{palette quaternary}{fg=black,bg=black!10}
\setbeamercolor*{structure}{fg=black}
\setbeamerfont{frametitle}{shape=\scshape}
\setbeamerfont{institute}{shape=\scshape}
\setbeamerfont{section in head/foot}{shape=\scshape}
\setbeamerfont{subsection in head/foot}{shape=\scshape}
\setbeamertemplate{bibliography item}{}
\setbeamertemplate{itemize items}[ball]
\setbeamertemplate{navigation symbols}{}
\setbeamertemplate{footline}[frame number]
\usetikzlibrary{calc,arrows}
\graphicspath{{graphics/}{graphics/movies/}{graphics/images/}}
\usepackage{remreset}                  % hack to display beamer navigation
\makeatletter                          % circles even if not declaring
\@removefromreset{subsection}{section} % subsections
\makeatother                           % see: http://tex.stackexchange.com/a/2078
\setcounter{subsection}{1}             % see: https://bitbucket.org/rivanvx/beamer/issue/218

% `biblatex' configuration
\usepackage[backend=biber,
style=authortitle-comp,
]{biblatex}
\addbibresource{presentation.bib}

% `enumitem' configuration
\usepackage{enumitem}
\setlist[itemize,1]{label=\usebeamertemplate{itemize item}}
\setlist[itemize,2]{label=\usebeamertemplate{itemize subitem}}
\setlist[itemize,3]{label=\usebeamertemplate{itemize subsubitem}}
\DeclareMathOperator{\sinc}{sinc}


% `graphicx' configuration
\usepackage{graphicx}
\begin{document}
	\title{Nonlinear Solvers}
	%\subtitle{MATH-151:  Mathematical Algorithms in Matlab}
	\author{MATH-151:  Mathematical Algorithms in Matlab}
	\date[202X]{October 9, 2023}
	\titlegraphic{\includegraphics[height = 3cm]{CSU_Ram_Logo.jpg}}



%%%%%%%%%%%%%%%%%%%%%%%%%%%%%%%%%%%%%%%%%%%%%%%%%%%%%%

\begin{frame}
\titlepage
\end{frame}
%%%%%%%%%%%%%%%%%%%%%%%%%%%%%%%%%%%%%%%%%%%%%%%%%%%%%%%%%
\section{Nonlinear Equations}

\begin{frame}{Why Solve Nonlinear Equations?}
	\footnotesize
	\begin{itemize}
		\item We are going to focus now on finding $x$ for a function such that $f(x) = 0$
		\item Lines are great, we know how to solve for where a line crosses zero quite easily, but many things in real life are nonlinear so we need to develop techniques for finding zeros for those equations as well
		\item Why do we want to find zeros in the first place?
		\begin{itemize}
			\item If we know how to find where $f(x)=0$, we know how to find where it is any other $c$ since we can instead solve $f(x) - c = g(x) = 0$
			\item If we want to find where two curves $f(x)$ and $g(x)$ intersect, we can do $f(x) - g(x) = h(x) = 0$
			\item Most engineering problems are interested in optimization. We remember from Calc I, if we want to find a minimizer or maxmizer, we start by finding a critical point where $f'(x) = 0$
		\end{itemize}
		\item Basically any time we want to know where anything occurs, we either run into a line or a case to solve for the root of a nonlinear equation!
		\begin{itemize}
			\item We usually like to know when/where things occur
		\end{itemize}
	\end{itemize}
\end{frame}


%%%%%%%%%%%%%%%%%%%%%%%%%%%%%%%%%%%%%%%%%%%%%%%%%%%%%%%%%

\begin{frame}{Bisection Method}
	\footnotesize
	\begin{itemize}
		\item We've already seen one of these a few weeks ago! The Bisection method!
		\item Suppose we know $f(a)$ is negative and $f(b)$ is positive. If $f$ is a continuous function, it must pass through 0 between $a$ and $b$.
		\item So we cut it in half and try $f(m)$, where $m  =\frac{a+b}{2}$
		\begin{itemize}
			\item If it has the same sign as $f(a)$, $m$ becomes our new $a$
			\item If it has the same sign as $f(b)$, $m$ becomes our new $b$
		\end{itemize} 
		\begin{center}
			\includegraphics[width = 3cm]{bisection_1.png} \hspace{0.25cm}
			\includegraphics[width = 3cm]{bisection_2.png} \hspace{0.25cm}
			\includegraphics[width = 3cm]{bisection_3.png}
		\end{center}
		\item This gives us a new, smaller possible range, so we do Bisection Method on that range
		\item We continue until $f(m)$ is sufficiently small.
	\end{itemize}
\end{frame}

%%%%%%%%%%%%%%%%%%%%%%%%%%%%%%%%%%%%%%%%%%%%%%%%%%%%%%%%%

\section{Fixed Point Methods}

\begin{frame}{Fixed Point Equations}
	\footnotesize
	\begin{itemize}
		\item How do we tell the computer to know when it's found a zero? For that we introduce the concept of a \textbf{fixed point}. 
		\item If $g(x)$ is a function that tells us our next guess for the zero, a fixed point satisfies the following
		\begin{equation*}
			x_n = g(x_n) = x_n + f(x_n)h(x_n)
		\end{equation*}
		\item When this occurs, our guess for the next point is the same as our current guess
		\begin{itemize}
			\item This will occur when $f(x_n) = 0$
		\end{itemize}
		\item That's what we are looking for! We found an $x_n$ such that $f(x_n)=0$.
		\begin{itemize}
			\item Remember, we will almost never get exactly 0 so we repeat our process until $f(x_n)$ or $x_n - x_{n-1}$ become small enough
			\item Or perhaps, we repeat \textit{while} $f(x_n)$ is too big
		\end{itemize}
		\item Great! So what can we do to set these up? What would this $g(x_n)$ look like?
	\end{itemize}
\end{frame}
%%%%%%%%%%%%%%%%%%%%%%%%%%%%%%%%%%%%%%%%%%%%%%%%%%%%%%%%%

\begin{frame}{Newton's Method}
	\footnotesize
	\begin{itemize}
		\item This is a widely used method that is usually much faster than bisection method!
		\item We pretend our function is the tangent line at our current guess, so that means our next guess for $x$ is where the tangent line crosses 0
		\begin{equation*}
			x_{n+1} = x_n - \frac{f(x_n)}{f'(x_n)}
		\end{equation*}
		\begin{center}
			\includegraphics[width = 3cm]{newton_1.png} \hspace{0.25cm}
			\includegraphics[width = 3cm]{newton_2.png} \hspace{0.25cm}
			\includegraphics[width = 3cm]{newton_3.png}
		\end{center}
		\item We are not only using the values of our function, but also information about what the curve looks like. This will help us get closer to our 0 faster!
		\begin{itemize}
			\item This does have some risk though, we could encounter a point where $f'(x_n) = 0$...
		\end{itemize}
	\end{itemize}
	
\end{frame}

%%%%%%%%%%%%%%%%%%%%%%%%%%%%%%%%%%%%%%%%%%%%%%%%%%%%%%%%%

\begin{frame}{Secant Method}
	\footnotesize
	\begin{itemize}
		\item Sometimes we don't have a derivative function, or it would be very difficult to implement \item In those cases, we can estimate our derivative using finite difference method with our last 2 points!
		\begin{itemize}
			\item But this means we need to make 2 initial guesses to start
		\end{itemize}
		\item Doing this we have a slightly different method called \textbf{Secant Method}
		\begin{equation*}
			x_{n+1} = x_n - f(x_n)\frac{x_n - x_{n-1}}{f(x_n) - f(x_{n-1})}
		\end{equation*}
		\begin{center}
			\includegraphics[width = 3cm]{secant_1.png} \hspace{0.25cm}
			\includegraphics[width = 3cm]{secant_2.png} \hspace{0.25cm}
			\includegraphics[width = 3cm]{secant_3.png}
		\end{center}
		\item The code I use in my research is actually using Secant method because our gradients are too complicated to find analytically!
	\end{itemize}
\end{frame}
\end{document}